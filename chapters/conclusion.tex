\chapter{Conclusion}
\label{chap:conclusion}

The theoretical part discussed the basic principles of full-text search engines and their differences compared to relational database management systems. This part also introduced the main technologies used for the development of the EEG/ERP Portal.

The practical part
In Chapter \ref{chap:analysis}, the main requirements were set. Also the introduced search engines and libraries were compared. Based on the comparison and the requirements, the Lucene-based Solr search server was chosen as the the best solution upon which is built in the next chapters.

In Chapter \ref{chapter:indexDesign}, indexable data were identified and an appropriate index structure was designed.

% Praci je mozno dale rozsirit. Implementovane reseni umoznuje indexovat dalsi zdrojova data, napr. 
The results of this thesis can be extended both in phases of indexing and searching. 
In the first case, the implementation is open to indexing other data sources, such as data from Facebook or Twitter.

The created search functionality can be later worked on

% dat sem i later work?