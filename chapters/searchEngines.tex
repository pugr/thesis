\chapter{Available Search Engines}
\label{chap:engines}

% http://www.sigir.org/forum/2012D/p095.pdf?searchterm=lucene

% http://stackoverflow.com/questions/737275/comparison-of-full-text-search-engine-lucene-sphinx-postgresql-mysql

% http://php.vrana.cz/fulltextove-vyhledavani-sphinx.php

% http://beerpla.net/2009/09/03/comparison-between-solr-and-sphinx-search-servers-solr-vs-sphinx-fight/

According to the comparisons of fulltext search engines in \cite{MiddletonBaeza} and \cite{SinghSearchEngines}, there are several reasonable options to choose from. 
Unfortunately, there is not enough comparative benchmarks on performance of search engines. Available comparisons are often not so objective since, for example, only the default settings of search engines are used in the tests or just one use case is applied
(as in \cite{SinghSearchEngines}). 
This may lead to false conclusions obtained from these benchmarks because search engines' capabilities might not have been fully demonstrated and tested. 
It is also worth mentioning that some comparisons were made a few years ago and are likely to be out-dated since these technologies evolve quite quickly.
All these experiments showed, however, that database fulltext search possibilities are way too slow \cite{BenchmarkLuceneRelDB,BenchmarkMysqlLuceneSphinx}.

In the following paragraphs the reader can grasp a basic overview on several full text search engines which are considered to be fairly
performant and usable for common full text search scenarios.


\section{Indri}

Indri \cite{IndriHome} is an academic C++ based text search engine
developed at the University of Massachusetts and is a part of the
Lemur Project. The engine is interesting because of its implementation
as it combines inference networks with language modeling. Its API
is accessible also from other languages such as Java or C\#. Great
support of scaling and support of true multithreaded operation, enabling
concurrent adding, querying and deleting documents belong to its main
features. In the technical paper \cite{ComparisonLuceneIndri}
it was shown that Indri is very performant and in comparison with
Lucene it achieves even better results in terms of retrieval effectiveness
for short queries, index size and performance.


\section{Lucene}

Lucene \cite{LuceneHome} is an open source Java-based search library. 
It can enrich applications by its ability to index data and search over them. 
These two main items form what is commonly reffered to as Lucene core. 
Apart from the core, its latest version comprises useful features related to full text search problems (e.g. result highlighting). 
Lucene is highly modularized and its API enables a user to extend its functionality relatively easily \cite{McCandless:2010:LAS:1893016}. 

It stores its index as files on disk. When the search is made, segments of indexes are copied to memory, thus its memory consumption is high and mostly stable.

According to \cite{MiddletonBaeza,ComparisonLuceneIndri}, it belongs to the fastest engines and its performance shines especially
when querying one- or two-word phrases. 
Its small size of index it creates is also a plus when the lack of memory could be an issue.
Thanks to its portability and its active development and active and numerous community, it has become the leading open source search engine used in many successful projects \cite{LuceneWhoUses}.

According to its official documentation \cite{LuceneScoring}, 
\textit{``Lucene scoring uses a combination of the Vector Space Model (VSM) of Information Retrieval and the Boolean model''}.
The Boolean model is used to filter the documents that are then scored.

The current version of Lucene, Lucene 4.0, \textit{,,represents a significant milestone in the development of Lucene due to a number of new features and efficiency improvements as compared to previous versions of Lucene.``} \cite{ApacheLucene4}


\section{Sphinx}

Sphinx \cite{SphinxHome} is an open source search server distributed under GPL license. 
It consists of an indexing tool, which is also referred to as indexer, and a searching daemon. 
Sphinx is written in C++ and is especially designed for indexing database systems (it integrates well with MySQL).
The reason behind its connection with MySQL RDBMS is to enable efficient full-text search on a large amount of database data \cite{aksyonoff2011introduction}.
It allows a user to issue queries with SQL-like syntax for indexing the database content and searching in the index.
Besides database indexing, it also supports an XML format for arbitrary data. 
Considering results from all found benchmarks involving Sphinx \cite{IndriHome,MiddletonBaeza,BenchmarkLuceneSphinxNewer,BenchmarkMysqlLuceneSphinx}, its indexing speed is quick and its search speed belongs to the fastest ones. 
It offers API bindings for several platforms/languages, including Java. 
Among its strengths we can name quick installation and deployment and a perfect online documentation for an open source project. 
Extensibility is quite an issue if our application requires additional features
\cite{esSphinxLuceneSolXapianWhich}. 

\section{Xapian}

According to information found on its official website \cite{XapianHome}, it is an open-source search library written in C++ based on a probabilistic retrieval model. 
It provides a user with bindings that allow to use it from a couple of other languages, including Java, C\# or Ruby. 
Its index size is generally larger when compared to other search engines, but this extra piece of information contained in the index allows Xapian to delete or update a document in a more correct way. 
It is done by storing a list of elements in each document in the termlist table\cite{XapianIndexSize}. 
The benchmarks in \cite{SinghSearchEngines} from 2009 show that its search performance is said to be equally good as the one of Lucene.

\section{Zettair}

Another search engine that received a very positive rating in \cite{MiddletonBaeza} is Zettair \cite{ZettairHome}. 
The comparison in this paper concludes that Zettair has the fastest indexer, provides fast querying speeds and has also the best retrieval effectiveness on average. 
Zettair is being developed at the Australian university and is written entirely in C. 
One of its main features is the ability to handle large data sets (100 GB and more) due to its scalability to large collections \cite{ZettairHome}.

Among its disadvantages belong its inability to index nothing else than text and html and the fact that the index cannot be built in
parallel using multiple machines. 
There are currently no ports to other languages available, so the full text search, when using Zettair, must be implemented in C.


\section{Lucene-based Search Solutions}
\label{sec:luceneBasedSolutions}

Popularity of Lucene is reflected in the existence of several search solutions which are based on this search library.
Because Lucene itself as a search library provides the core functionality, i.e. indexing and searching documents, its direct integration in applications without any additional enhancements can be cumbersome in the most common use cases.
Unless there are special requirements involving access to low-level APIs of Lucene, opting for the already proved search solutions is recommended \cite{Solr3EnterpriseSS}.

The solutions built on top of Lucene can be divided into \textit{search tools} and \textit{search servers}. 
Search tools extend Lucene in a specific way to cover the problem domain and can be embedded to an application in a form of a library. 
They are usually dependent on other technologies which must be included in the target application. 
A widely used example of such search tool is \textit{Hibernate Search}.
Search servers, on the other hand, are fully independent standalone applications which communicate with the application typically by using the HTTP protocol. The representatives of this category are \textit{Solr} and \textit{ElasticSearch}.


\subsubsection{Hibernate Search}

Hibernate Search \cite{HibernateSearchHome} is an enterprise search tool that forms a bridge between Hibernate and Lucene. 
The aim of Hibernate Search is to enable full text search over persistent domain model by overcoming certain mismatches between the domain model and simple Lucene documents.
In order to make Hibernate Search work, Hibernate or JPA must be used as an object-relational mapping between Java objects and database tables. 
% It leverages the power of Lucene for indexing and searching.
% and of Solr (more about Solr can be found in section 3.6.2), whose of useful features and analyzers can be used. 
It manipulates with persisted Hibernate entities and makes them searchable by adding them in the Lucene index. 
By default, Hibernate Search automatically updates Lucene index after an object is saved or updated by using Hibernate.
In order to achieve indexing of Hibernate entities, created model classes must be enriched of specific Hibernate Search annotations for full text search purposes.
Apart from annotating the entities that should be indexed, no extra configuration is needed. 


\subsubsection{Solr}

Solr \cite{SolrHome} is an open source search server built on top of Lucene which runs within a servlet container such as Jetty or Tomcat.
Because it can be considered as a Lucene extension, most of the Lucene terminology is used for Solr as well. 
Unlike Lucene, document fields have to be defined in the application schema file.

Solr extends Lucene by providing many useful features related to full text search, e.g. hit highlighting,
faceted search, rich document handling or the did-you-mean feature, just to name a few. 
Since Solr runs as a separate process, it communicates with applications via HTTP requests, which represent query data, and HTTP responses, representing search results found in the index, by exposing its REST-like API. 
This technology is configuration-driven and in some cases everything can be set up with no need of writing a single line of Java code. 


\subsubsection{ElasticSearch}

Elasticsearch \cite{ElasticSearchHome} is another promising young technology build on Lucene, designed from its early beginnings as a highly scalable solution suitable for big data.

Unlike Solr, it provides a more flexible approach of data definition.
Elasticsearch is a one-man project with a steadily growing community, but compared to Solr, it is relatively immature in terms of some of its features \cite{ElasticSearchComparisonSolr, ElasticSearchComparisonSolrAnother}.