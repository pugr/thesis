\chapter{Testing}

This chapter presents created test cases to prove the correct functionality of the implementation of the full text search for EEG/ERP Portal. The tests are of two types. The first type of the tests are JUnit tests which cover the functionality on the core, application logic level. The second test category includes Selenium tests whose purpose is to simulate user interaction with the full text search.

All tests interact with an independently running Solr test server to access indexed data. The server runs at .... on port 8686.

All tests use the test database and an instance of the Solr server which is devoted only for testing purposes.
Although the tests could have been performed in combination with the embedded version of the Solr server, its remotely running instance imitates the real situation better.

%Jako reseni se pro testovani nabizela take tzv. embedded verze Solr serveru (trida EmbeddedSolrServer).
%Protoze je server spojen s aplikaci, HTTP pristup je v tomto pripade
%jen simulovan, navic neumoznuje takove moznosti spravy a nastaveni
%jako samostatne bezici server. Proto byla dana prednost
%samostatne instanci Solru, ktera verohodneji imituje provoz
%na ostrem Solr serveru.


\section{Unit Tests}

There were several test cases created by using JUnit4. It was possible to integrate JUnit with Spring in order to use declared Spring beans in the test cases. 
Their main purpose is to verify expected functionality of various classes related to the full text search.


\section{UI Tests (Selenium)}

These tests were created to simulate actions of users using the full text search of EEG/ERP Portal. 

