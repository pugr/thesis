
\chapter{Introduction}

Accessing required information from a large set of data in a quick
and user-friendly manner is no longer an unachievable goal. Advancements
in the field of information retrieval in the last few decades made
this requirement a common and in many cases also an essential part
of many tasks of everyday's life in a modern society. This work deals
with the topic of efficient full text search over data belonging to
the domain of the EEG/ERP Portal, a piece of software developed at
the University of West Bohemia in Pilsen. 

The work is organized into two main parts. The first part, theoretical
part, is included in chapters 2 - 4 and covers most of the theoretical
knowledge about various topics used later throughout the thesis. Chapter 2 introduces
core concepts of information retrieval (IR), comparison of IR and
database systems, finally, several available open source full text
search engines are described here as well. In Chapter 3 EEG/ERP Portal
together with its underlying technologies corrently used for its development
are presented. 

...TODO

The practical part builds upon the theoretical aspects.

Chapter 5 describes the problem to be solved and the specific requirements.
Chapter 6 discusses the architecture that was chosen based on these
requirements and the implementation of the full text search section.
Chapter 7 comprises discussion about the chosen implementation and
potential future work based on results of this thesis.

Chapter 8 finally summarizes the contents of the thesis.