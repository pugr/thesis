\usepackage{hyphenat}
\usepackage[pdftex]{graphicx}
\usepackage{pslatex}
\usepackage{comment}
\usepackage[english]{babel}
\usepackage[scaled=1.0]{helvet}
\usepackage[utf8]{inputenc}
\usepackage{blindtext}
%\usepackage{styles/minted}
\usepackage{subfig}
\usepackage[justification=centering]{caption}
\usepackage{url}
\usepackage{titlesec}
\usepackage[T1]{fontenc}
\usepackage{lmodern} % vypada s tim fakt cool
\usepackage{epstopdf}
\usepackage{microtype}
\usepackage{color}
\usepackage{pifont}

\usepackage{titling}

\usepackage{listings}
\usepackage[labelfont=bf]{caption}
\usepackage{courier}

\lstset{
         basicstyle=\footnotesize\ttfamily, % Standardschrift
         numbers=left,           	    % Ort der Zeilennummern
         numberstyle=\footnotesize,          % Stil der Zeilennummern
         stepnumber=1,                   % the step between two line-numbers. If it is 1 each line will be numbered
				 numbersep=10pt,                  % how far the line-numbers are from the code
         tabsize=2,                  % Groesse von Tabs
         extendedchars=true,         %
         breaklines=true,            % Zeilen werden Umgebrochen
				 prebreak=\mbox{\tiny$\hookleftarrow$},
         %keywordstyle=\color{black}\textbf,
				 keywordstyle=\color{black}\ttfamily,
    		 frame=tb,         
 %        keywordstyle=[1]\textbf,    % Stil der Keywords
 %        keywordstyle=[2]\textbf,    %
 %        keywordstyle=[3]\textbf,    %
 %        keywordstyle=[4]\textbf,   \sqrt{\sqrt{}} %
         stringstyle=\color{black}\ttfamily, % Farbe der String
         showspaces=false,           % Leerzeichen anzeigen ?
         showtabs=false,             % Tabs anzeigen ?
         %xleftmargin=17pt,
         %framexleftmargin=17pt,
         %framexrightmargin=5pt,
         %framexbottommargin=4pt,
         backgroundcolor=\color[rgb]{0.93,0.93,0.93},  
         showstringspaces=false      % Leerzeichen in Strings anzeigen ?        
 }
 \lstloadlanguages{% Check Dokumentation for further languages ...
         %[Visual]Basic
         %Pascal
         %C
         %C++
         %XML
         %HTML
         Java
 }
    %\DeclareCaptionFont{blue}{\color{blue}} 

  %\captionsetup[lstlisting]{singlelinecheck=false, labelfont={blue}, textfont={blue}}
  
\DeclareCaptionFont{blackas}{\color{black}}
\DeclareCaptionFont{black}{\color{black}\textbf}
% \DeclareCaptionFormat{listing}{\colorbox[cmyk]{0.43, 0.35, 0.35,0.01}{\parbox{\textwidth}{\hspace{15pt}#1#2#3}}}
\captionsetup[lstlisting]{labelfont=black, singlelinecheck=false, margin=0pt}

%\lstset{ %
%language=XML,                % choose the language of the code
%basicstyle=\small\ttfamily,        % the size of the fonts that are used for the code
%basewidth=0.51em,
%stringstyle=\ttfamily\color[rgb]{0.227,0.226,0.341},
%numbers=left,                   % where to put the line-numbers
%numberstyle=\footnotesize,      % the size of the fonts that are used for the line-numbers
%stepnumber=1,                   % the step between two line-numbers. If it is 1 each line will be numbered
%numbersep=10pt,                  % how far the line-numbers are from the code
%backgroundcolor=\color[rgb]{0.8,0.8,0.8},  % choose the background color. You must add \usepackage{color}
%showspaces=false,               % show spaces adding particular underscores
%showstringspaces=false,         % underline spaces within strings
%showtabs=false,                 % show tabs within strings adding particular underscores
%frame=single,           % adds a frame around the code
%tabsize=2,          % sets default tabsize to 2 spaces
%captionpos=b,           % sets the caption-position to bottom
%columns=fullflexible,
%aboveskip={1.5\baselineskip},
%breaklines=true,        % sets automatic line breaking
%breakatwhitespace=false,    % sets if automatic breaks should only happen at whitespace
%escapeinside={\%*}{*)}          % if you want to add a comment within your code
%}

