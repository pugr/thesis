\usepackage{hyphenat}
\usepackage[pdftex]{graphicx}
\usepackage{pslatex}
\usepackage{comment}
\usepackage[scaled=1.0]{helvet}
\usepackage[utf8]{inputenc}
\usepackage{blindtext}
%\usepackage{styles/minted}
\usepackage{subfig}
\usepackage[justification=centering]{caption}
\usepackage{url}
\usepackage{titlesec}
\usepackage[T1]{fontenc}
\usepackage{epstopdf}
\usepackage{microtype}
\usepackage{color}
\usepackage{pifont}

\usepackage{listings}
\lstset{ %
language=XML,                % choose the language of the code
basicstyle=\small\ttfamily,        % the size of the fonts that are used for the code
basewidth=0.51em,
stringstyle=\ttfamily\color[rgb]{0.727,0.626,0.941},
numbers=left,                   % where to put the line-numbers
numberstyle=\footnotesize,      % the size of the fonts that are used for the line-numbers
stepnumber=1,                   % the step between two line-numbers. If it is 1 each line will be numbered
numbersep=10pt,                  % how far the line-numbers are from the code
backgroundcolor=\color{white},  % choose the background color. You must add \usepackage{color}
showspaces=false,               % show spaces adding particular underscores
showstringspaces=false,         % underline spaces within strings
showtabs=false,                 % show tabs within strings adding particular underscores
frame=single,           % adds a frame around the code
tabsize=2,          % sets default tabsize to 2 spaces
captionpos=b,           % sets the caption-position to bottom
columns=fullflexible,
aboveskip={1.5\baselineskip},
breaklines=true,        % sets automatic line breaking
breakatwhitespace=false,    % sets if automatic breaks should only happen at whitespace
escapeinside={\%*}{*)}          % if you want to add a comment within your code
}

